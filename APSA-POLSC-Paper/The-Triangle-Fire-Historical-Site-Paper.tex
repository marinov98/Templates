\documentclass[12pt]{article}
\renewcommand\thesection{}
\usepackage[letterpaper, portrait, margin=1in]{geometry}
\usepackage[american]{babel}
\usepackage[style=apa]{biblatex}
\DeclareLanguageMapping{american}{american-apa}
\defbibheading{bibliography}[\bibname]{\section{#1}}
\addbibresource{\jobname.bib}
\usepackage{hyperref} % load this (last) if you want live links
\usepackage{comment}
\usepackage{csquotes}
\usepackage{indentfirst}
\usepackage{setspace}
\doublespacing

\usepackage{sectsty}

\usepackage{titlesec}
\titleformat{\section}
  {\normalfont\fontsize{14}{15}\itshape}{\thesection}{0em}{}
  %{\normalfont\fontsize{12}{15}\bfseries}{\thesection}{1em}{}

% \allsectionsfont{\centering}

\usepackage{etoolbox} % single space and small block quotes
\AtBeginEnvironment{quote}{\singlespacing\small}

\begin{document}

  \title{The Triangle Fire and its Repercussions \\
  {\large \textit {A Microcosm of the 20th Century American Labor Movement}}}
  \author{Ralph Vente}
  \date{April 16, 2018}

  \begin{titlepage}
    \maketitle
    \hrule
    \thispagestyle{empty}

    \begin{abstract}

      If you go there now, the Brown Building is a part of New York University.
      It was donated in 1929 by Frederick Brown who gave it the name we know it
      by today. If you went there on the 25th of March, 1911, you would have
      witnessed a fire erupt there in what proved to be one of the deadliest
      industrial disasters in American History. The public outcry and
      legislative fervor cemented New York as a exemplar for progressivism
      around the nation. It was an integral component of the American Labor
      Movement, and its ramifications can still be felt throughout all of
      working society today. In this paper, I will examine how the fire and the
      movement it ignited the Twentieth Century American Labor movement and
      where workers of the United States stand today because of their sacrifice.

    \end{abstract}
  \end{titlepage}

\section{March 25th, 1911}

It was around 4:30 pm on Saturday, March 25th, 1911. This was decades before the
Fair Labor Standards Act would guarantee that these workers had the weekends off
for leisure. The Triangle Shirtwaist Factory was just a few minutes from closing
for the day. By 4:40, fire had broken out on the eighth floor of the ten floor
factory building \parencite[628]{Ev1995}. The first alarm was called in at 4:45
\parencite[628]{Ev1995}.

To this point, the building was described as ``fireproof'' because of its steel
and concrete construction. That description would prove dangerously inaccurate.
Within 30 minutes, the fire had consumed three floors. 141 people died that day
as reported by the New York Times the next day. Historians today agree on the
figure 146 \parencite[622]{Ev1995}. Some of the women had jumped out of the
windows, filling the air with screaming on their way down
\parencite[14]{stein2010triangle}. Others had suffocated, trapped inside the
building with locked doors that were opened too late. Others still were overcome
by the flames and burned to death \parencite{shepherd1911eyewitness}.

The victims consisted mostly of young, uneducated immigrant women
\parencite{martineztriangle}. These were a particularly vulnerable subset of
workers because they had to face the language barrier and lacked the social
capital of native born Americans.


% this was before women had the right to vote unionizing was one of few options
% to cement their position as people who had rights.


\section{Foreshadowing: Before the Fire}

% Historical Context what was legal, what wasn't
% child labor was still legal
% trusts, robber barons etc

After hearing about this, a common reaction is ``Why did no one anticipate
this?'' The answer is they did. The Triangle Shirtwaist Company had resisted
implementing changes from trade unions and state legislation that would have
improved worker safety. The Triangle Fire only happened the way it did because
of the plethora of code violations committed by the Factory's administrators.
Clear policy adherence would have saved lives. Among the enormous number of
negligent acts, the owners' mismanagement of the number of stairways, the
cramped layout of the factory floor, and the locked doors all cost 123 women and
23 men to lose their lives that day \parencite{von2004triangle}.

This wasn't even the first time that the word ``fire'' was on the table. The New
York City Fire Commissioner had certified the building as a ``fire trap'' a mere
3 months prior \parencite[627]{Ev1995}. City regulations already required that
buildings that tall should have 3 staircases to the roof. The builders only
included a single stairway that could be used for escape on its own, a single
point of failure. To save costs, they counted the fire escape as a stairwell,
and even then, it only went down to the second floor. The same fire escape was
reported as ``dangerously loose'' by the City Fire Commissioner. It came as no
surprise then, that it collapsed under the weight of so many trying to escape.
The final stairway was partial and didn't connect the ground to the roof, but
stopped at the 10th floor \parencite[627]{Ev1995}.

The New York Labor Code required 250 cubic feet of air per worker so that
workers could have enough space to breathe. The company circumvented this
precaution by making ceilings high. The result? At face value, workers seemed to
be getting their respective airspace. In practice, however, conditions were
grim. Arthur F. McEvoy, doctor in US Economic History, notes that long lines of
sewing machines stretched from wall to wall, separated by narrow aisles where
workers say. Together, this made a formation that he describes as ``maze''
\parencite[628]{Ev1995}.

Finally, according to the same document by Dr. McEvoy, the Labor Code also
required that doors be unlocked during the day and that doors open outwards into
the stairwell. Not only did the stairway doors open inward into the factory
floor, but were they also locked \parencite[628]{Ev1995}, yet another obstacle
that contributed to so many deaths that day. Their excuse? They alleged that the
company was only trying to prevent theft of company products by preventing
discreet departures through an unsupervised exit. \parencite{Ev1995}. In other
words, company administrators jeopardized so many lives that day expressly to
protect their own interests.

The atrocious working conditions didn't end there or start there. These same
garment factory workers had, in the past, faced working conditions that they
fought ruthlessly. For example, before 1910, these and other garment workers had
to work for 70 or more hours a week without overtime pay, earning a measly 6
dollars \parencite{von2004triangle}. According to the Consumer Price Index, that
would be the equivalent of approximately 167 dollars a week in 2018.

In response, the 400 workers of Local 25 of the International Ladies' Garment
Workers' Union (ILGWU) decided to go on strike. While recovering from a beating
from hired arms, union activist Clara Lemlich convinced garment workers
across the city to participate in a general strike. In late September, 20,000
workers walked off their jobs. By the 8th of February, the employers at the
Triangle Factory finally ran out of scab labor
% footnote this (workers hired to fill in during a strike or labor dispute)
and gave in to most of the demands of the strikers. They won a 20 percent raise,
a reduced workweek of 52 hours, and overtime pay \parencite{von2004triangle}.
They also called on city leaders to institute and enforce better worker health
and safety standards, but these provisions wouldn't come in time to lessen
casualties on the day of the fire.

% widen scope
% explain how this was not an anomaly
% explain the ongoing clash between trade unionists and industry owners


% https://awitous.wordpress.com/2015/04/21/the-triangle-shirtwaist-factory-fire/

\section{The Fallout: Public Solidarity}

In its wake, the Triangle Fire left the public astonished, and the worker safety
movement caught the national eye, but no amount of attention could prevent the
factory owners from escaping unscathed. They were never found guilty. On December
27th, they were neither found to have ordered the doors to be locked nor
found to have had any idea that managers were doing it. In the end, the families of the
workers were compensated a mere 75 dollars each, roughly 1900 dollars in 2018,
equivalent to five weeks' pay at the new hard-won wage rate
\parencite[9]{robinson2018crimes}. The public outcry could still be felt as
angry crowds yelled ``Murderers!'' as they left the courthouse
\parencite[9]{robinson2018crimes}.

On April 5th, however, the public showed its unity. Vast stretches of the
sidewalk were covered as a total of 300,000 people lined the streets to watch
120,000 in a funeral march to honor the victims \parencite[644]{Ev1995}.

\section{The Legacy: Legislative Reform}

Garment workers in New York had already undertaken two major strikes. First
there was the ``Uprising of 20,000'' mentioned before. Then, there was another
strike in New York of 60,000 workers. In Chicago, there was a 1910 -- 1911 strike 40,000
workers strong \parencite{WorkingClass}. By the time the Factory Fire
happened, the labor movement was already in full swing. The fire intensified
their efforts. Garment workers from the ILGWU and other various unions formed
the Amalgamated Clothing Workers of America (ACWA), and together, the tens of
thousands of members went on strike. They earned higher wages, shorter hours, 
and better working conditions for themselves. But more than that, the public
momentum led to the creation of a bill which went on to establish the New York
State Factory Investigating Commission on June 30th of the same year
\parencite{Ev1995}.

The number 146 pales in comparison to the astronomical 35,000 workers who died
in work accidents annually at the start of the 1900s \parencite{WorkingClass}.
It was clear any true progress would have to adress the general problem of
workplace safety. The Committee was made up of union activists, as well as experts
in fire engineering and architecture \parencite[5]{martineztriangle}. As McEvoy
notes,

\begin{quote}
By 1915 the Factory Commission had engineered the passage of 36 new statuses on
safety regulation, hours limitation, child labor, and so on \dots In 1912 New
York amended its constitution so as to permit enforcement of a workers'
compensation statute and instituted the new scheme the following year \dots
Spurred by public outrage, the nation's leading industrial state revolutionized
its political economy in the space of three years. The Triangle fire, then,
marked a historical discontinuity of some significance. It suddenly made
palpable a new social order, awareness of which had been gathering for decades
\parencite[646]{Ev1995}.
\end{quote}

The change didn't stop there. 1913 saw the creation of the Bureau of Fire
Prevention, which created legislation such as the Line Safety Code (1913). This
``Focused on providing people a path for `prompt escape' in case of fire by
identifying and creating standards to counter fire hazards''
\parencite[7]{martineztriangle}. All things considered, Labor's response to the
Triangle Fire had effects that panned the entire US. It set the stage for
the Progressive Era and spawned New York styled labor reform on a national
scale \parencite{Ev1995}. 

\section{The Building Today}

According to its website, in 1929, philanthropist Frederick Brown, who had
purchased the building prior, donated it to New York University, which still
owns it today. When I visited the Brown Building, I was expecting a bit more
commemoration. When I got there, there was little more than a few plaques noting
its historical significance.

As for unions, their membership declined after the 1920s. Strikes became less
frequent, and ``welfare capitalism'' took its footing. Here, employers
offered certain benefits along with the traditional wage or salary system
\parencite{WorkingClass}. Such offerings included home loans, group
insurance policies, and stock options. Together with Progressive era reforms,
and a transition into the service economy we know today,
this reduced workers' reliance on unions to maintain a healthy standard of
living.

Despite spiking again to 26 percent in the 1930s \parencite{WorkingClass}, union
membership slowly slumped again to its modern figure of 10 percent
\parencite{indexes2018bureau}. Even though labor unions and public solidarity have
seemingly fallen to all time lows, workers in the US are generally better off
today due to the progressive labor reform spearheaded by workers in the 20th
century. Whether unions are relevant in the US today is up for contention, but
it wouldn't be a stretch to say we won't soon forget the tragedy that sparked such an influential
movement.

% https://en.wikipedia.org/wiki/2013_Savar_building_collapse
% https://www.historyonthenet.com/authentichistory/1898-1913/2-progressivism/3-laborreform/3-trianglefire/index.html
% Presentation I'll pardon you some of the more vivid imagry 
  
%  \parencite[25]{Ev1995}
%  \parencite[25]{Co2011}
% This is an in text citation: \textcite{Ar95} This is a parenthetical one: \parencite[25]{En1995}.
%
% Who? List and describe the significant participants at the historical site.
% When? When the event happen? Or in what time period did the event happen? What

% Use 12-point font with one-inch margins, page numbers centered at the bottom of
% the page, and one staple in the upper left corner. Proper citations and
% references must be used following the American Political Science Association
% Style Manual, which can be found at
% http://www.apsanet.org/Portals/54/APSA%20Files/publications/APSAStyleManual2006.pdf
%
% relevance to modern politics:

%  * developing countries still have to deal with this

  \newpage
  \printbibliography

\end{document}
